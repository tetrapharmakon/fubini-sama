\documentclass[a4paper,10pt]{amsart}

\input{pream/prambolo}

\newcommand{\cmts}{\mathrel{\text{\copyright}}}

\author{Ivan di Liberti \and Fosco Loregian}
\title{On some properties preserved by Yoneda extension}
\begin{document}
\maketitle
\begin{abstract}

\end{abstract}
\begin{itemize}
	\item Intro (a few technical lemma)
	\item Vecchio \cite{adamek2002classification}
	\item coVecchio
	\item General problem
	\item faithfullezza, conservativity
\end{itemize}
\begin{definition}[The $\cmts$ relation]
Let $I,J$ two small categories; we define a relation $\cmts$ on $\Cat$ saying that $I\cmts J$ if and only $I$-colimits commute with $J$-limits over $\Set$, \ie for any functor $F : I\times J \to \Set$ there is a canonical isomorphism
\[\textstyle
\colim_I \lim_J F \cong \lim_J \colim_I F
\]
\end{definition}
\begin{remark}[Elementary observations on $\cmts$]
As it is always the case, the relation $\cmts$ on $\Cat$ generates a Galois connection on $2^{\Cat}$ given by intersection:
\[
\sfX^{\cmts} := \{J\in\Cat\mid I\cmts J\; \forall I\in\sfX\}
\]
and dually, we can define ${}^{\cmts}\sfY$. This is a (contravariant) Galois connection in the sense that ${}^{\cmts}(\firstblank)\dashv (\firstblank)^{\cmts}$: the usual properties of an adjunction induce a closure and an interior operator (given by unit and counit) in such a way that $\sfX\subseteq ({}^{\cmts}\sfX)^{\cmts}$ and $\sfX\subseteq {}^{\cmts}(\sfX^{\cmts})$ for every $\sfX\subseteq \Cat$; 
\end{remark}
\begin{definition}[doctrine system]
We define a \emph{doctrine system} as a pair $(\sfX,\sfY)$ such that $\sfX={}^{\cmts}\sfY$ and $\sfY = \sfX^{\cmts}$. Obviously, a doctrine system is uniquely determined by the specification of its left or right class.
\end{definition}
\begin{example}
We can define the following doctrine systems on $\Cat$:
\begin{enumerate}
	\item $(\textsf{Set}_{<\alpha},\textsf{Conn}_{<\alpha})$;
	\item $(\alpha\textsf{-Filt},\textsf{Cat}_{<\alpha})$;
	\item $(\alpha\textsf{-Sift},\textsf{Set}_{<\alpha})$;
	\item \dots
\end{enumerate}
\end{example}
\bibliography{allofthem}{}
\bibliographystyle{amsalpha}
\hrulefill 
\end{document}